\chapter{PMD Rule Definitions}
\label{apx:pmd}

\begin{lstlisting}[caption={Utility class which attempts to extract the required information about method calls from the Abstract Syntax Tree.}, captionpos=b, label=lst:pmd_util, numbers=left, showstringspaces=false]
public class Util {
  public static class MethodCall {
    public final String targetOwner;
    public final Class<?> targetOwnerClass;
    public final String target;
    public final int argumentCount;
    public final Node source;

    public MethodCall(Class<?> targetOwner,
        JavaNameOccurrence occurrence) {
      this.targetOwner = targetOwner == null ? null :
        targetOwner.getCanonicalName();
      this.targetOwnerClass = targetOwner;
      this.target = occurrence.getImage();
      this.argumentCount = occurrence.getArgumentCount();
      this.source = occurrence.getLocation();
    }
  }

  public static List<MethodCall> getMethodCallsFrom(JavaNode body){
    List<MethodCall> methodCalls = new ArrayList<>();

    // Find expressions that contain at least one method call
    Set<List<JavaNameOccurrence>> invocationChains = body
        .findDescendantsOfType(ASTPrimaryExpression.class).stream()
        .map(expr -> new NameFinder(expr).getNames())
        .filter(names ->
          names.stream().anyMatch(name ->
            name.isMethodOrConstructorInvocation()
          )
        )
        .collect(Collectors.toSet());

    for (List<JavaNameOccurrence> chain : invocationChains) {
      Class<?> targetOwner;
      if (chain.get(0).isMethodOrConstructorInvocation()) {
        // First name in chain is a method call

        // Look for previous sibling
        JavaNode node = chain.get(0).getLocation();
        if (node.getIndexInParent() > 0) {
          // Get return type from sibling
          Class<?> siblingType = getType(
            node.getParent().getChild(
              node.getIndexInParent() - 1
            )
          );
          if (siblingType != null) {
            MethodCall call = new MethodCall(
              siblingType, chain.get(0)
            );
            methodCalls.add(call);
          }
        } else {
          // Should be local method, super method or static import
          // Assumed to be local method
          ASTClassOrInterfaceDeclaration enclosingClass = body
            .getFirstParentOfType(
              ASTClassOrInterfaceDeclaration.class
            );
          if (enclosingClass != null
              && enclosingClass.getType() != null) {
            MethodCall call = new MethodCall(
              enclosingClass.getType(), chain.get(0)
            );
            methodCalls.add(call);
          }
        }
      }

      // Resolve type of first sub-expression, i.e. owner of next target method
      targetOwner = getType(chain.get(0), body.getScope());

      // Iterate over suffixes to resolve method calls
      for (Iterator<JavaNameOccurrence> it = chain.listIterator(1); it.hasNext();) {
        JavaNameOccurrence occurrence = it.next();

        if (occurrence.isMethodOrConstructorInvocation()
            && targetOwner != null) {
          methodCalls.add(new MethodCall(
            targetOwner, occurrence
          ));
        }

        // Return type becomes target owner for the next iteration
        targetOwner = getType(occurrence.getLocation());
      }
    }

    return methodCalls;
  }

  public static Class<?> getType(NameOccurrence nameOccurrence,
      Scope scope) {
    Class<?> type = null;

    // Search for name in current and parent scopes
    while (scope != null) {
      Optional<NameDeclaration> nameDeclaration = scope
        .getDeclarations().keySet().stream()
        .filter(name -> name.getName() != null)
        .filter(name -> 
          name.getName().equals(nameOccurrence.getImage())
        )
        .findFirst();

      if (nameDeclaration.isPresent()) {
        type = getType(nameDeclaration.get().getNode());
        break;
      }

      scope = scope.getParent();
    }

    if (type == null) {
      // Try to extract type directly from node
      type = getType(nameOccurrence.getLocation());
    }

    return type;
  }

  public static Class<?> getType(ScopedNode node) {
    if (node instanceof TypeNode) {
      return ((TypeNode) node).getType();
    }

    return null;
  }
}
\end{lstlisting}

\begin{lstlisting}[caption={Constraint 1.}, captionpos=b, label=lst:pmd_1, numbers=left, showstringspaces=false]
public class LogSecurityEventsRule extends AbstractJavaRule {
  private static final String LOGGER =
    "edu.ncsu.csc.itrust.action.EventLoggingAction";
  private static final Collection<String> SECURITY_SERVICES =
    Arrays.asList(
      "edu.ncsu.csc.itrust.action.ActivityFeedAction",
    );

  public LogSecurityEventsRule() {
    super();

    // Types of nodes to visit
    addRuleChainVisit(ASTMethodDeclaration.class);
  }

  @Override
  public Object visit(ASTMethodDeclaration method, Object data) {
    ASTClassOrInterfaceDeclaration owningClass = method
      .getFirstParentOfType(ASTClassOrInterfaceDeclaration.class);
    if (owningClass == null) {
      // Method defined in enum => ignore
      return data;
    }

    if (!SECURITY_SERVICES.contains(owningClass.getBinaryName())
        || !method.isPublic()) {
      // Not a security event; skip method
      return data;
    }

    boolean callsLogger = Util.getMethodCallsFrom(method).stream()
        .anyMatch(call -> LOGGER.equals(call.targetOwner));

    if (!callsLogger) {
      addViolation(data, method);
    }

    return data;
  }
}
\end{lstlisting}

\begin{lstlisting}[caption={Constraint 2.}, captionpos=b, label=lst:pmd_2, numbers=left, showstringspaces=false]
public class AuthSingleComponentRule extends AbstractJavaRule {
  private static final String AUTH_POINT =
    "atm.transaction.Transaction";
  private static final String AUTH_ENFORCER =
    "atm.transaction.Transaction";

  public AuthSingleComponentRule() {
    super();

    // Types of nodes to visit
    addRuleChainVisit(ASTClassOrInterfaceDeclaration.class);
  }

  @Override
  public Object visit(ASTClassOrInterfaceDeclaration clazz,
      Object data) {
    boolean isAuthPoint = AUTH_POINT.equals(clazz.getBinaryName());
    Stream<Util.MethodCall> methodCallsFromClass = clazz
      .findDescendantsOfType(
        ASTClassOrInterfaceBodyDeclaration.class
      )
      .stream()
      .flatMap(body -> Util.getMethodCallsFrom(body).stream());

    if (isAuthPoint) {
      // Ensure at least one call to authentication enforcer
      boolean callsEnforcer = methodCallsFromClass
        .anyMatch(call -> AUTH_ENFORCER.equals(call.targetOwner));

      if (!callsEnforcer) {
        addViolationWithMessage(data, clazz, "#2 Authentication point must call authentication enforcer");
      }
    } else {
      // Ensure no calls to authentication enforcer
      methodCallsFromClass
        .filter(call -> AUTH_ENFORCER.equals(call.targetOwner))
        .forEach(offendingCall -> {
          String message = "#2 Method invocation to enforcer must be performed at auth point";
          addViolationWithMessage(data, offendingCall.source, message);
        });
    }

    return data;
  }
}

\end{lstlisting}

\begin{lstlisting}[caption={Constraint 3.}, captionpos=b, label=lst:pmd_3, numbers=left, showstringspaces=false]
class CentralSendingRule extends AbstractJavaRule {
  private static final String SENDING_POINT =
    "atm.transaction.Transaction";
  private static final Collection<String> SENDERS =
    Arrays.asList(
      "atm.physical.NetworkToBank"
    );

  public CentralSendingRule() {
    super();

    // Types of nodes to visit
    addRuleChainVisit(ASTClassOrInterfaceBodyDeclaration.class);
  }

  @Override
  public Object visit(
      ASTClassOrInterfaceBodyDeclaration body,
      Object data) {
    ASTClassOrInterfaceDeclaration owningClass =
      body.getFirstParentOfType(
        ASTClassOrInterfaceDeclaration.class
      );
    boolean isSendingPoint = owningClass != null
      && SENDING_POINT.equals(owningClass.getBinaryName());
    if (isSendingPoint) {
      // Method defined in sending point; skip this method
      return data;
    }

    Util.getMethodCallsFrom(body).stream()
      .filter(call ->
        SENDERS.contains(call.targetOwner)
        && call.argumentCount > 0
      )
      .forEach(offendingCall -> 
        addViolation(data, offendingCall.source)
      );

    return data;
  }
}
\end{lstlisting}

\begin{lstlisting}[caption={AnnotationHelper, a class that is used to store annotations to, and fetch annotations from, a text file. A separate file is created for each annotation, and each line in this file describes the canonical name of a class or method that is marked with the annotation. This functionality is required to implement case 2 of constraint 4, i.e. looking up the annotations of a remote method or class.}, captionpos=b, label=lst:pmd_annotation_helper, numbers=left, showstringspaces=false]
public class AnnotationHelper {
  public static void createFiles(List<String> annotations) {
    for (String annotation : annotations) {
      try {
        Files.write(
          Paths.get("pmd_" + annotation + ".txt"),
          new byte[0],
          StandardOpenOption.CREATE,
          StandardOpenOption.TRUNCATE_EXISTING
        );
      } catch (IOException e) {
        e.printStackTrace();
      }
    }
  }

  public static void dumpAnnotation(String annotation, JavaNode node) {
    try {
      String content = getLocation(node) + "\n";

      Files.write(getPath(annotation), content.getBytes(),
        StandardOpenOption.APPEND);
    } catch (IOException e) {
      e.printStackTrace();
    }
  }

  public static List<String> getAnnotations(String annotation) {
    try {
      String dump = new String(
        Files.readAllBytes(getPath(annotation))
      );
      return Arrays.asList(dump.strip().split("\n"));
    } catch (IOException e) {
      e.printStackTrace();
    }

    return Collections.emptyList();
  }

  private static Path getPath(String annotation) {
    return Paths.get("pmd_" + annotation + ".txt");
  }

  private static String getLocation(JavaNode node) {
    String location;
    if (node instanceof ASTClassOrInterfaceDeclaration) {
      location = ((ASTClassOrInterfaceDeclaration) node)
        .getBinaryName();
    } else {
      location = node.getFirstParentOfType(
          ASTClassOrInterfaceDeclaration.class
        )
        .getBinaryName();
    }

    if (node instanceof ASTMethodDeclaration) {
      ASTMethodDeclaration method = (ASTMethodDeclaration) node;
      location += "." + method.getName();
    } else if (node instanceof ASTConstructorDeclaration) {
      location += "." + node.getImage();
    }

    return location;
  }
}
\end{lstlisting}

\begin{lstlisting}[caption={AnnotationDumper, a rule that traverses all classes and dumps their annotations into a text file using AnnotationHelper.}, captionpos=b, label=lst:pmd_annotation_dumper, numbers=left, showstringspaces=false]
public class DumpAnnotations extends AbstractJavaRule {
  private static final List<String> ANNOTATIONS = Arrays.asList(
    "com.github.secarchunit.concepts.UserInput",
    "com.github.secarchunit.concepts.InputValidator"
  );

  public DumpAnnotations() {
    super();

    // Create annotation dump files
    AnnotationHelper.createFiles(ANNOTATIONS);

    // Types of nodes to visit
    addRuleChainVisit(ASTClassOrInterfaceDeclaration.class);
  }

  @Override
  public Object visit(ASTClassOrInterfaceDeclaration clazz,
      Object data) {
    for (String annotation : ANNOTATIONS) {
      if (clazz.isAnnotationPresent(annotation)) {
        AnnotationHelper.dumpAnnotation(annotation, clazz);
      }
    }

    for (ASTMethodDeclaration method : clazz
        .findDescendantsOfType(ASTMethodDeclaration.class)) {
      for (String annotation : ANNOTATIONS) {
        if (method.isAnnotationPresent(annotation)) {
          AnnotationHelper.dumpAnnotation(annotation, method);
        }
      }
    }

    for (ASTConstructorDeclaration constructor : clazz
        .findDescendantsOfType(ASTConstructorDeclaration.class)) {
      for (String annotation : ANNOTATIONS) {
        if (constructor.isAnnotationPresent(annotation)) {
          AnnotationHelper.dumpAnnotation(annotation, constructor);
        }
      }
    }

    return super.visit(clazz, data);
  }
}
\end{lstlisting}

\begin{lstlisting}[caption={Constraint 4.}, captionpos=b, label=lst:pmd_4, numbers=left, showstringspaces=false]
public class ValidateInputRule extends AbstractJavaRule {
  private static final String USER_INPUT =
    "com.github.secarchunit.concepts.UserInput";
  private static final String INPUT_VALIDATOR =
    "com.github.secarchunit.concepts.InputValidator";

  public ValidateInputRule() {
    super();

    // Types of nodes to visit
    addRuleChainVisit(ASTMethodDeclaration.class);
  }

  @Override
  public Object visit(ASTMethodDeclaration method, Object data) {
    if (method.isAnnotationPresent(USER_INPUT)) {
      // Check for in-line validation (case 1)
      if (method.isAnnotationPresent(INPUT_VALIDATOR)) {
        // Checks out
        return data;
      }

      // Check for calls to validator (case 2)
      List<String> inputValidators = AnnotationHelper
        .getAnnotations(INPUT_VALIDATOR);
      boolean callsValidator = Util.getMethodCallsFrom(method)
        .stream()
        .anyMatch(call ->
          inputValidators.contains(call.targetOwner)
          || inputValidators.contains(
            call.targetOwner + "." + call.target
          )
        );

      if (callsValidator) {
        // Checks out
        return data;
      }

      // Check if all callers of this method are input validators (case 3)
      // This can't be done in PMD => assume no validation occurs

      addViolation(data, method);
    }

    return data;
  }
}

\end{lstlisting}

\begin{lstlisting}[caption={Constraint 5.}, captionpos=b, label=lst:pmd_5, numbers=left, showstringspaces=false]
public class RestrictThreadSpawningRule extends AbstractJavaRule {
  private static final String RESOURCE_RESTRICTION =
    "com.github.secarchunit.concepts.ResourceRestriction";

  public RestrictThreadSpawningRule() {
    super();

    // Types of nodes to visit
    addRuleChainVisit(ASTMethodDeclaration.class);
    addRuleChainVisit(ASTConstructorDeclaration.class);
  }

  @Override
  public Object visit(ASTMethodDeclaration method, Object data) {
    if (containsViolation(method)) {
      addViolation(data, method);
    }

    return super.visit(method, data);
  }

  @Override
  public Object visit(ASTConstructorDeclaration constructor,
      Object data) {
    if (containsViolation(constructor)) {
      addViolation(data, constructor);
    }

    return super.visit(constructor, data);
  }

  private boolean containsViolation(Annotatable annotatable) {
    boolean startsThreadOrProcess =
      Util.getMethodCallsFrom(annotatable).stream()
      .anyMatch(call ->
        Thread.class.isAssignableFrom(call.targetOwnerClass)
          && "start".equals(call.target)
        || ProcessBuilder.class.equals(call.targetOwnerClass)
          && "start".equals(call.target)
        || Runtime.class.equals(call.targetOwnerClass)
          && "exec".equals(call.target)
      );

    if (!startsThreadOrProcess) {
      // Skip
      return false;
    }

    // Ensure there is resource restriction
    if (annotatable.isAnnotationPresent(RESOURCE_RESTRICTION)) {
      // Checks out
      return false;
    }

    ASTClassOrInterfaceDeclaration sourceOwner =
      annotatable.getFirstParentOfType(
        ASTClassOrInterfaceDeclaration.class
      );
    if (sourceOwner.isAnnotationPresent(RESOURCE_RESTRICTION)) {
      // Checks out
      return false;
    }

    // Violation
    return true;
  }
}
\end{lstlisting}