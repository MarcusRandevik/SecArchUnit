% CREATED BY DAVID FRISK, 2018
\chapter{Conclusion}
We presented an investigation and empirical evaluation of using and extending the ArchUnit framework to support the validation of security architectural constraints and compared it to SonarQube and PMD. The analysis covered both the ability to reliability detect violations of constraints as well as the architectural appropriateness. We based on a set of constraints gathered from architectural patterns, rules, and weaknesses. The constraints were applied to three open-source systems, the largest containing over 28k lloc. Our results show that while all tools were able to perform similarly in their ability to detect a subset of the composed violations, SecArchUnit was the only one with the ability to detect violations concerning the flow of information.  SecArchUnit also provides a more suitable interface for the architect as it builds a model of the entire system allowing analysis on the dependencies across several classes. In comparison, both SonarQube and PMD analyze each class individually, making analysis of the relation between classes significantly more difficult. 

We are hopeful that SecArchUnit could provide architects, in particular those in agile projects, with the ability to specify and enforce security architectural constraints using a semantic that is familiar to the source code of the developed system. Future research should evaluate if the usage of SecArchUnit decreases architectural violations over time and increases productivity compared to that of manual reviews. 

