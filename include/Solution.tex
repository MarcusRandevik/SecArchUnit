\chapter{Enforcing Constraints}
\todo{implementation of the constraints is very much a work in progress}

This chapter explains how the constraints can be expressed and validated with the tool. The constraints are divided into three distinct categories. The first category contains the constraints that are possible to express in ArchUnit as-is. The second category describes constraints that are enforceable with the help of additional information in source code. The third and final category details constraints that require an extension of ArchUnit to be possible to enforce.

\section{Support in ArchUnit as-is}

ArchUnit contains an extensive vocabulary for expressing typical architectural constraints. These constraints are typically composed of three parts. The first part indicates the type of Java construct that should be inspected. These constructs include classes, methods, fields and constructors. The second part contains a predicate that selects a subset of these constructs. The third part defines the condition that must hold true for all the selected constructs.

An example of a rule defined solely using this standard vocabulary can be seen in Listing \ref{lst:standard_vocabulary}, where each of the three aforementioned parts of the constraint has been separated into their own line. The rule is a simple example of complete mediation, where some internal classes must only be accessed through a mediator.

\begin{lstlisting}[caption={Example of a rule that is expressed with the standard vocabulary.}, captionpos=b, label=lst:standard_vocabulary, numbers=left]
ArchRule rule = classes()
    .that().resideInAPackage("..internal..")
    .should().onlyBeAccessed().byAnyPackage("..mediator..");
\end{lstlisting}

In cases where this vocabulary is not sufficient for expressing a constraint, there is a possibility to define custom predicates and conditions over any given construct and supplying these as arguments to the \texttt{that()} and \texttt{should()} methods.

\subsection{Constraints}

\subsubsection*{Log all security events with relevant information}
This constraint is expressed with the assumption that there is a central class through which security related actions are performed. Any publicly accessible method in this class must contain at least one call to the logging facility. Both the class that represents the secure service facade and the class responsible for logging are expressed in the constraint itself, meaning there is no need for additional information in the source code.

\subsubsection*{Enforce authentication/ authorization at a single point in the system}
...

\subsubsection*{Every outbound message is sent from a central point of the system}
...

\section{With Additional Information in Code}

Some of the architectural constraints require that the developer injects additional information into the source code.

In some cases, this information is simply an indicator that says something about an entire class. Naming the class with a specific suffix is one approach to accomplish this. Another approach is to implement an empty interface, which is the technique used with Java's \texttt{Serializable}\footnote{https://docs.oracle.com/javase/7/docs/api/java/io/Serializable.html} interface. 

In other cases, however, the information may be required for methods of arbitrary signatures and even specific fields. For the purposes of flexibility and minimal obtrusiveness, any extra information is expressed in the form of annotations. These can be applied to classes, fields, methods and parameters without changing the underlying architecture of the system.

\subsection{Constraints}

\subsubsection*{Input from a user is validated}
...

\subsubsection*{Allocation of resources is limited or throttled}
While resources is a broad term, this constraint focuses on preventing the exhaustion of CPU and memory resources through the creation of new threads and processes. As such, every block of code that contains a call to the \texttt{start()} method of a \texttt{Thread}\footnote{https://docs.oracle.com/javase/7/docs/api/java/lang/Thread.html} or any of its subclasses, must be marked as containing a resource restriction mechanism. The same rule is applied for calls to \texttt{ProcessBuilder.start()}\footnote{https://docs.oracle.com/javase/7/docs/api/java/lang/Process.html} and \texttt{Runtime.exec()}\footnotemark[3], which lead to the creation of new processes.
% TODO manual footnote, adjust as necessary

The marking is done with the help of an annotation, either on the relevant method or the entire class. The decision of how the restriction mechanism is implemented is left to the developer of the system.

\section{With Extensions to ArchUnit}

In the current ArchUnit API, a rule that aims to constrain access to a method (or field) must be expressed in terms of the type signatures of the source and target methods. Some of our constraints require knowledge about the type signature of the object that is being passed as a parameter. This is a non-issue when fields and method parameters are of the same types as the objects being passed to them. However, in cases where a method signature accepts a "more general" type, such as an \texttt{Object}, there is no way for ArchUnit to constrain the types of the objects that are actually being passed as parameters.

ArchUnit builds its representation of the architecture using ASM\footnote{https://asm.ow2.io/}, a Java bytecode analysis framework. This framework contains functionality for keeping track of the stack and local variables while analyzing the instructions of a method. With knowledge of the type signatures of the references on the stack at the time of a method call or field assignment, it is possible to determine the type signatures of objects passed as parameters or an object being assigned to a field. Our extension provides this additional information in ArchUnit's representation of accesses to fields and methods, which the rule definitions can then make use of.

\subsection{Constraints}

\subsubsection*{Do not allow sensitive information to bleed to other components}
...

\subsubsection*{Do not log secrets}
...
