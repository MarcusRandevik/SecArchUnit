\chapter{Selection of Architectural Security Constraints}

This chapter describes the methodology of collecting a final list of security constraints from three sources. 

\section{Collection of Security Measures}

The processed collection of security measures can be seen in Table~\ref{tab:all_measures}.
\todo{Fix sub-goals: some of them are anti-goals currently}
\todo{Reformulate everything as constraints}
% E.g. #5, what is "it"/the subject? Who strips the data?
% Some of these are just guidelines, not actual constraints.

% (2020-04-10) Subgoals do not seem to add any value as we have not used them when considering the implementation. 
% (2020-04-10) Previous entry 16 and 17 were merged in table 4.1 instead of 4.2 as they were previously not grouped due to sub-goal differences

\begin{table}
\begin{tabular}{lp{10cm}ll}
\hline
\textbf{\#} & \textbf{Constraint} & \textbf{Goal} \\
\hline
1  & Exceptions shown to the client must be sent to a sanitizer \checkmark & Confidentiality \\
\rowcolor{RowColor}
2  & Sensitive information must not bleed to components with lower security classification \checkmark & Confidentiality \\
3  & Sensitive information must be encrypted before transmission \checkmark & Confidentiality \\
\rowcolor{RowColor}
4  & Every outbound message must be sent from a single component responsible for transmissions \checkmark & Confidentiality \\
5  & Data that passes a trust boundary must first be sent to a component responsible for hiding or removing sensitive data \checkmark & Confidentiality \\
\rowcolor{RowColor}
6  & Secrets must not be exposed in log messages \checkmark & Confidentiality \\
7  & The system must not provide functionality to decrypt secured log messages \checkmark & Confidentiality \\
\rowcolor{RowColor}
8  & Output passing between components must be validated against its specification & Integrity \\
9  & Input from a user must pass through a component validating the data \checkmark & Integrity \\
\rowcolor{RowColor}
10 & The session object must not be accessible to the user \checkmark & Integrity \\
11 & Components must store its state as restorable checkpoints \checkmark & Availability \\
\rowcolor{RowColor}
12 & Spawning of threads must be limited or throttled \checkmark & Availability \\
13 & The system must not have multiple points of access \checkmark & Accountability \\
\rowcolor{RowColor}
14 & At least one checkpoint must be initialized after successful authentication and authorization \checkmark & Accountability \\
15 & Methods related to security events must call the logger \checkmark & Accountability \\
\rowcolor{RowColor}
16 & Authentication and authorization must each be enforced in a single component \checkmark & Accountability \\
17 & Security relevant log messages must be encrypted and immutable & Accountability \\
\hline
\end{tabular}
\caption{Security constraints and their related CIAA goals.}
\label{tab:all_measures}
\end{table}

\section{Final Selection}

As explained in Section~\ref{sec:limitations}, the aim is not to demonstrate the enforceability of as many constraints as possible, but rather to investigate the feasibility of using the tool in this manner. To that end, a subset of the full list of security constraints is selected for enforcement. The final list contains 7 architectural security constraints, as this allows us to cover at least one constraint from each goal. The selected constraints can be seen in Table~\ref{tab:selected_measures}. The remainder of this section presents each selected constraint in further detail.

\begin{table}
\begin{tabular}{llp{11cm}}
\hline
\textbf{\#} & \textbf{ID in 4.1} & \textbf{Constraint} \\
1 & 15 & Security events are logged with appropriate information\\
\rowcolor{RowColor}
2 & 16,17 & Enforce authentication/ authorization at a single point in the system\\
3 & 4 & Every outbound message is sent from a central point of the system\\
\rowcolor{RowColor}
4 & 9 & Input from a user is validated\\
5 & 12 & Allocation of resources is limited or throttled\\
\rowcolor{RowColor}
6 & 2 & Do not allow sensitive information to bleed to other components\\
7 & 6 & Do not log secrets\\
\hline
\end{tabular}
\caption{Constraints that have been selected for enforcement.}
\label{tab:selected_measures}
\end{table}

\todo{Present each of the final constraints, see notes in comment}
% For each constraint:
% * describe what it means
% * typical way to enforce it
% * which source it comes from (literature)
% * maybe describing a security attack scenario that this constraint aims to avoid

\subsection{Log all security events} \todo{Marcus}

\subsection{Enforce authentication at single point} \todo{Marcus}

\subsection{Messages are sent from central point} \todo{Marcus}

\subsection{User input is validated}  \todo{Marcus/ Patrik}

\subsection{Usage of resources is restricted} \todo{Patrik}

\subsection{Sensitive information must stay within trust boundary}\todo{Patrik}

\subsection{Secrets must not be exposed in log messages}\todo{Patrik}

