\chapter{SecArchUnit Code Excerpts}
\begin{lstlisting}[caption=Constraint 4: performDirectOrIndirectValidation custom condition., captionpos=b, label=lst:constraint_4_condition, numbers=left, showstringspaces=false]
ArchCondition<JavaCodeUnit> performDirectOrIndirectValidation =
new ArchCondition<>("perform direct or indirect validation") {
  @Override
  void check(JavaCodeUnit codeUnit, ConditionEvents events) {
    if (codeUnit.isAnnotatedWith(InputValidator.class)) {
      // Validates input directly => condition passed
      return;
    }

    boolean callsValidator = codeUnit.getCallsFromSelf().stream()
      .map(call -> call.getTarget())
      .anyMatch(target -> 
        target.isAnnotatedWith(InputValidator.class)
        || target.getOwner().isAnnotatedWith(InputValidator.class)
      );
    if (callsValidator) {
      // Calls a validator => condition passed
      return;
    }

    boolean calledAtLeastOnce = !codeUnit.getAccessesToSelf()
      .isEmpty();
    boolean onlyCalledByValidators = codeUnit.getAccessesToSelf()
      .stream()
      .map(call -> call.getOrigin())
      .allMatch(origin -> origin.isAnnotatedWith(InputValidator.class)
        || origin.getOwner().isAnnotatedWith(InputValidator.class));
    if (calledAtLeastOnce && onlyCalledByValidators) {
      // Is only called by validators => condition passed
      return;
    }

    String message = codeUnit.getFullName() + " takes user input that is never validated";
    events.add(SimpleConditionEvent.violated(codeUnit, message));
  }
};
\end{lstlisting}

\begin{lstlisting}[caption=Constraint 5: \texttt{aThreadIsStartedWithoutRestriction} custom predicate., captionpos=b, label=lst:constraint_5_predicate_1, numbers=left, showstringspaces=false]
DescribedPredicate<JavaMethodCall> aThreadIsStartedWithoutRestriction =
new DescribedPredicate<>("a thread is started") {
    @Override
    public boolean apply(JavaMethodCall call) {
        AccessTarget.MethodCallTarget target = call.getTarget();

        boolean isRestricted = call.getOrigin()
            .isAnnotatedWith(ResourceRestriction.class);
        boolean startsAThread = 
            target.getOwner().isAssignableTo(Thread.class)
            && target.getName().equals("start");

        return !isRestricted && startsAThread;
    }
};
\end{lstlisting}

\begin{lstlisting}[caption=Constraint 5: \texttt{aProcessIsStartedWithoutRestriction} custom predicate., captionpos=b, label=lst:constraint_5_predicate_2, numbers=left, showstringspaces=false]
DescribedPredicate<JavaMethodCall> aProcessIsStartedWithoutRestriction =
new DescribedPredicate<>("a process is started") {
    @Override
    public boolean apply(JavaMethodCall call) {
        AccessTarget.MethodCallTarget target = call.getTarget();

        boolean isRestricted = call.getOrigin()
            .isAnnotatedWith(ResourceRestriction.class);
        boolean startsAProcess =
            target.getOwner().isEquivalentTo(ProcessBuilder.class)
                && target.getName().equals("start")
            || target.getOwner().isEquivalentTo(Runtime.class)
                && target.getName().equals("exec");

        return !isRestricted && startsAProcess;
    }
};
\end{lstlisting}

\clearpage
\begin{lstlisting}[caption=Constraint 6: \texttt{notBleedToInsecureComponents} custom condition., captionpos=b, label=lst:constraint_6_condition, numbers=left, showstringspaces=false]
ArchCondition<JavaField> notBleedToInsecureComponents =
new ArchCondition<>("not bleed to insecure components") {
    @Override
    public void check(JavaField field, ConditionEvents events) {
        // Direct access
        field.getAccessesToSelf().stream()
            .filter(access -> !access.getOriginOwner()
                .isAnnotatedWith(AssetHandler.class))
            .forEach(offendingFieldAccess -> {
                String message = offendingFieldAccess
                    + ": access to asset " + field.getName();
                events.add(SimpleConditionEvent.violated(offendingFieldAccess, message));
            });

        // Access via getter method
        field.getAccessesToSelf().stream()
            .filter(access -> access.getOrigin() instanceof JavaMethod)
            .map(access -> (JavaMethod) access.getOrigin())
            .filter(method ->
                method.getReturnValueHints().stream()
                    .anyMatch(hint ->
                        field.equals(hint.getMemberOrigin())
                    )
                )
            .flatMap(method -> method.getCallsOfSelf().stream())
            .filter(call -> !call.getOriginOwner()
                .isAnnotatedWith(AssetHandler.class))
            .forEach(offendingMethodCall -> {
                String message = offendingMethodCall
                    + ": access to asset " + field.getName()
                    + " (via getter method)";
                events.add(SimpleConditionEvent.violated(offendingMethodCall, message));
            });
    }
};
\end{lstlisting}

\clearpage
\begin{lstlisting}[caption=Constraint 7: \texttt{passSecretArgumentTo} custom condition., captionpos=b, label=lst:constraint_7_condition, numbers=left, showstringspaces=false]
ArchCondition<JavaClass> passSecretArgumentTo(
    DescribedPredicate<JavaAccess<?>> target) {
return new ArchCondition<>("pass @Secret argument to "
    + target.getDescription()) {
  @Override
  public void check(JavaClass clazz, ConditionEvents events) {
    clazz.getMethodCallsFromSelf().stream()
      .filter(call -> target.apply(call))
      .forEach(callToTarget -> {
        InformationFlow.recurseOnHints(
            callToTarget.getArgumentHints()
          )
          .filter(hint -> hint.getMemberOrigin() != null)
          .map(hint -> hint.getMemberOrigin())
          .filter(member ->
            member.isAnnotatedWith(Secret.class)
            || member.getOwner().isAnnotatedWith(Secret.class)
          )
          .distinct()
          .forEach(secretMember -> {
            String message = callToTarget.getSourceCodeLocation()
              + " passes secret "
              + secretMember.getOwner().getSimpleName()
              + "." + secretMember.getName();
            events.add(SimpleConditionEvent.satisfied(
              callToTarget,
              message)
            );
          });
      });
  }
};
\end{lstlisting}

\clearpage
\begin{lstlisting}[caption=Constraint 7: \texttt{InformationFlow} class used for hint recursion., captionpos=b, label=lst:constraint_7_flow, numbers=left, showstringspaces=false]
class InformationFlow {
    Stream<Hint> recurseOnHints(Set<Hint> hints) {
        return recurseOnHints(hints, 5).distinct();
    }

    Stream<Hint> recurseOnHints(Set<Hint> hints, int depth) {
        if (depth == 0 || hints.isEmpty()) {
            return hints.stream();
        }

        // Hints with an originating member
        Set<JavaMember> hintOrigins = hints.stream()
            .filter(hint -> hint.getMemberOrigin() != null)
            .map(hint -> hint.getMemberOrigin())
            .collect(Collectors.toSet());

        // Hints flowing into a field
        Stream<Hint> hintsFlowingIntoFields = hintOrigins.stream()
            .filter(member -> member instanceof JavaField)
            .map(member -> (JavaField) member)
            .flatMap(hint -> hint.getAccessesToSelf().stream())
            .flatMap(access -> access.getArgumentHints().stream());

        // Hints flowing out of a method
        Stream<Hint> hintsFlowingOutOfMethods = hintOrigins.stream()
            .filter(member -> member instanceof JavaMethod)
            .map(member -> (JavaMethod) member)
            .flatMap(method -> method.getReturnValueHints().stream());

        // Collect hints from this level
        Set<Hint> recursedHints = Stream.concat(
                hintsFlowingIntoFields, 
                hintsFlowingOutOfMethods
            )
            .collect(Collectors.toSet());

        // Concatenate this level of hints with the next recursion level
        return Stream.concat(hints.stream(), recurseOnHints(recursedHints, depth - 1));
    }
}
\end{lstlisting}