\chapter{SonarQube Rule Definitions}
\label{apx:sonarqube}

\begin{lstlisting}[caption={Constraint 1.}, captionpos=b, label=lst:sq_1, numbers=left, showstringspaces=false]
public class LogSecurityEventsRule
    extends IssuableSubscriptionVisitor {

  public static String LOGGER_CLASS = "edu.ncsu.csc.itrust.action.EventLoggingAction";
  public static List<String> SECURITY_CLASSES = Arrays.asList(
      "ActivityFeedAction".toLowerCase(),
  );

  MethodMatchers loggerMethods = MethodMatchers.create()
    .ofTypes(LOGGER_CLASS).anyName().withAnyParameters()
    .build();

  @Override
  public List<Tree.Kind> nodesToVisit() {
    return Collections.singletonList(Tree.Kind.METHOD);
  }

  @Override
  public void visitNode(Tree tree) {
    MethodTree method = (MethodTree) tree;

    String methodEnclosingClass = method.symbol().enclosingClass()
      .name().toLowerCase();
    if (!SECURITY_CLASSES.contains(methodEnclosingClass)) {
      // Method not defined in a security service => skip
      return;
    }

    boolean isPublicMethod = false;
    for (ModifierKeywordTree keywordTree : method.modifiers().modifiers()) {
      if (keywordTree.modifier() == Modifier.PUBLIC) {
        isPublicMethod = true;
        break;
      }
    }

    if (!isPublicMethod) {
      // Not a public method => skip
      return;
    }

    boolean containsCallToLogger = false;

    ControlFlowGraph cfg = method.cfg();

    for (ControlFlowGraph.Block block : cfg.blocks()) {
      for (Tree blockTree : block.elements()) {
        if (blockTree.is(Tree.Kind.METHOD_INVOCATION)) {
          MethodInvocationTree mit = (MethodInvocationTree) blockTree;
          if (loggerMethods.matches(mit)) {
            // 
            containsCallToLogger = true;
          }
        }
      }
    }

    if (!containsCallToLogger) {
      reportIssue(method.simpleName(), "Secure classes must contain call to logger");
    }
  }
}
\end{lstlisting}

\begin{lstlisting}[caption={Constraint 2. This rule ensures that at least one method in the authentication point calls the authentication enforcer.}, captionpos=b, label=lst:sq_2, numbers=left, showstringspaces=false]
public class AuthSingleComponentRule
    extends IssuableSubscriptionVisitor {

  private static final String AUTH_POINT_CLASS =
    "OrderActionBean".toLowerCase();
  private static final String AUTH_ENFORCER_CLASS =
    "org.mybatis.jpetstore.web.actions.OrderActionBean";

  private final MethodMatchers authMethods =
    MethodMatchers.create()
    .ofTypes(AUTH_ENFORCER_CLASS).anyName().withAnyParameters()
    .build();

  private static final int INITIAL_AMOUNT_OF_METHODS = -1;
  private int methodsInAuthPoint = INITIAL_AMOUNT_OF_METHODS;
  private int methodsVisited = 0;
  private boolean containsCallToEnforcer = false;

  @Override
  public List<Tree.Kind> nodesToVisit() {
    return Collections.singletonList(Tree.Kind.METHOD);
  }

  @Override
  public void visitNode(Tree tree) {
    MethodTree method = (MethodTree) tree;

    String enclosingClassName = method.symbol().enclosingClass()
      .name().toLowerCase();
    // Check if we're in the auth point class
    if (!AUTH_POINT_CLASS.equals(enclosingClassName)) {
      return;
    }

    methodsVisited++;

    // If this is the first time we're in the auth point class, calculate the number of methods
    if (methodsInAuthPoint == INITIAL_AMOUNT_OF_METHODS) {
      Collection<Symbol> symbols = method.symbol().enclosingClass()
        .memberSymbols();
      setMethodsInAuthPoint(symbols);
    }

    // From the cgf, see if the methods contains a call to enforcer class
    ControlFlowGraph cfg = method.cfg();

    for (ControlFlowGraph.Block block : cfg.blocks()) {
      for (Tree blockTree : block.elements()) {
        if (blockTree.is(Tree.Kind.METHOD_INVOCATION)) {
          MethodInvocationTree mit =
            (MethodInvocationTree) blockTree;
          if (authMethods.matches(mit)) {
            containsCallToEnforcer = true;
          }
        }
      }
    }

    if (methodsVisited >= methodsInAuthPoint
        && !containsCallToEnforcer) {
      reportIssue(
        method.symbol().enclosingClass().declaration(),
        "Authpoint must contain call to enforcer"
      );
    }
  }

  /**
   * Calculate the amount of methods in the authpoint class
   * @param symbols the collection of symbols defined within the authpoint class
   */
  private void setMethodsInAuthPoint(Collection<Symbol> symbols) {
    for (Symbol symbol : symbols) {
      if (symbol.isMethodSymbol() 
          && !symbol.name().equals("<init>")) {
        if (methodsInAuthPoint == INITIAL_AMOUNT_OF_METHODS) {
          methodsInAuthPoint = 1;
        } else {
          methodsInAuthPoint++;
        }
      }
    }

    if (methodsInAuthPoint == INITIAL_AMOUNT_OF_METHODS) {
      methodsInAuthPoint = 0;
    }
  }
}
\end{lstlisting}

\begin{lstlisting}[caption={Constraint 2. This rule ensures that calls to the authentication enforcer only occur in the authentication point.}, captionpos=b, label=lst:sq_2b, numbers=left, showstringspaces=false]
public class AuthSingleComponentEnforcerRule
    extends IssuableSubscriptionVisitor {
  private static final String AUTH_POINT_CLASS =
    "OrderActionBean".toLowerCase();
  private static final String AUTH_ENFORCER_CLASS =
    "org.mybatis.jpetstore.web.actions.OrderActionBean";

  private final MethodMatchers enforcerMethods =
    MethodMatchers.create()
    .ofTypes(AUTH_ENFORCER_CLASS).anyName().withAnyParameters()
    .build();

  @Override
  public List<Tree.Kind> nodesToVisit() {
    return Collections.singletonList(Tree.Kind.METHOD_INVOCATION);
  }

  @Override
  public void visitNode(Tree tree) {
    MethodInvocationTree mit = (MethodInvocationTree) tree;

    if (enforcerMethods.matches(mit)) {
      // Get the class of the calling method
      Tree parent = mit.parent();
      while (!parent.is(Tree.Kind.CLASS)) {
        parent = parent.parent();
      }
      ClassTree classTree = (ClassTree) parent;
      String enclosingClassOfCaller = classTree.symbol()
        .name().toLowerCase();

      if (!AUTH_POINT_CLASS.equals(enclosingClassOfCaller)) {
        reportIssue(mit, "Method invocation to enforcer must be performed at auth points");
      }
    }
  }
}

\end{lstlisting}

\begin{lstlisting}[caption={Constraint 3.}, captionpos=b, label=lst:sq_3, numbers=left, showstringspaces=false]
class CentralMessageRule extends IssuableSubscriptionVisitor {
  private static final String SENDING_POINT = "Transaction";
  private static final MethodMatchers SENDERS =
    MethodMatchers.create()
    .ofTypes("atm.physical.NetworkToBank")
    .anyName()
    .addParametersMatcher(parameters -> !parameters.isEmpty())
    .build();

  @Override
  public List<Tree.Kind> nodesToVisit() {
    return Collections.singletonList(Tree.Kind.METHOD_INVOCATION);
  }

  @Override
  public void visitNode(Tree tree) {
    MethodInvocationTree methodInvocation =
      (MethodInvocationTree) tree;

    if (SENDERS.matches(methodInvocation)) {
      // Get class where method invocation takes place
      Tree parent = methodInvocation.parent();
      while (!parent.is(Tree.Kind.CLASS)) {
        parent = parent.parent();
      }
      ClassTree classTree = (ClassTree) parent;

      // Compare class with sending point
      String sendingClassName = classTree.symbol().name();
      if (!SENDING_POINT.equals(sendingClassName)) {
        reportIssue(methodInvocation, "Messages must only be sent from the sending point");
      }
    }
  }
}
\end{lstlisting}

\begin{lstlisting}[caption={Constraint 4.}, captionpos=b, label=lst:sq_4, numbers=left, showstringspaces=false]
public class ValidateUserInputRule
    extends IssuableSubscriptionVisitor {
  private static final String USER_INPUT =
    "com.github.secarchunit.concepts.UserInput";
  private static final String INPUT_VALIDATOR =
    "com.github.secarchunit.concepts.InputValidator";

  @Override
  public List<Tree.Kind> nodesToVisit() {
    return Collections.singletonList(Tree.Kind.METHOD);
  }

  @Override
  public void visitNode(Tree tree) {
    MethodTree methodTree = (MethodTree) tree;

    // Check if this method deals with user input
    if (!methodTree.symbol().metadata()
        .isAnnotatedWith(USER_INPUT)) {
      return;
    }

    // Check for in-line validation (case 1)
    if (methodTree.symbol().metadata()
        .isAnnotatedWith(INPUT_VALIDATOR)) {
      return;
    }

    // Check for calls to validator (case 2)
    for (Block block : methodTree.cfg().blocks()) {
      for (Tree blockTree : block.elements()) {
        if (blockTree.is(Tree.Kind.METHOD_INVOCATION)) {
          MethodInvocationTree mit = (MethodInvocationTree) blockTree;
          if (mit.symbol().metadata()
              .isAnnotatedWith(INPUT_VALIDATOR)
            || mit.symbol().enclosingClass().metadata()
              .isAnnotatedWith(INPUT_VALIDATOR)) {
            return;
          }
        }
      }
    }

    // Check if all callers are marked as validators (case 3)
    boolean hasCallers = !methodTree.symbol().usages().isEmpty();
    if (hasCallers) {
      boolean validatedInAllCallers = true;

      for (IdentifierTree caller : methodTree.symbol().usages()) {
        if (caller.symbol().metadata()
            .isAnnotatedWith(INPUT_VALIDATOR)) {
          // Caller method is validator
          continue;
        }

        if (caller.symbol().enclosingClass().metadata()
            .isAnnotatedWith(INPUT_VALIDATOR)) {
          // Caller class is validator
          continue;
        }

        validatedInAllCallers = false;
        break;
      }

      if (validatedInAllCallers) {
        return;
      }
    }

    reportIssue(methodTree, "User input must be validated");
  }
}
\end{lstlisting}

\begin{lstlisting}[caption={Constraint 5.}, captionpos=b, label=lst:sq_5, numbers=left, showstringspaces=false]
public class LimitThreadSpawnRule
    extends IssuableSubscriptionVisitor {
  private static final String RESOURCE_RESTRICTION =
    "com.github.secarchunit.concepts.ResourceRestriction";

  private final MethodMatchers threadStartMethods =
    MethodMatchers.create()
      .ofSubTypes("java.lang.Thread").names("start")
      .withAnyParameters().build();
  private final MethodMatchers processStartMethods =
    MethodMatchers.or(
      MethodMatchers.create()
        .ofTypes("java.lang.ProcessBuilder")
        .names("start").withAnyParameters()
        .build(),
      MethodMatchers.create()
        .ofTypes("java.lang.Runtime").names("exec")
        .withAnyParameters()
        .build()
    );

  @Override
  public List<Tree.Kind> nodesToVisit() {
    return Collections.singletonList(Tree.Kind.METHOD);
  }

  @Override
  public void visitNode(Tree tree) {
    MethodTree methodTree = (MethodTree) tree;

    if (methodTree.symbol().isAbstract()) {
      // Skip abstract methods, as they have no code block
      return;
    }

    if (methodStartsThreadOrProcess(methodTree)) {
      // Check for resource restriction marker...
      // ... on method
      if (methodTree.symbol().metadata()
          .isAnnotatedWith(RESOURCE_RESTRICTION)) {
        return;
      }

      // ... or on class
      if (methodTree.symbol().enclosingClass().metadata()
          .isAnnotatedWith(RESOURCE_RESTRICTION)) {
        return;
      }

      reportIssue(methodTree, "Thread spawns must be restricted");
    }
  }

  private boolean methodStartsThreadOrProcess(
      MethodTree methodTree) {
    for (Block block : methodTree.cfg().blocks()) {
      for (Tree blockTree : block.elements()) {
        if (blockTree.is(Tree.Kind.METHOD_INVOCATION)) {
          MethodInvocationTree mit =
            (MethodInvocationTree) blockTree;
          if (threadStartMethods.matches(mit)
              || processStartMethods.matches(mit)) {
            return true;
          }
        }
      }
    }
    return false;
  }
}
\end{lstlisting}